\documentclass[11pt,titlepage]{article}
\usepackage[margin=1in]{geometry}
\usepackage{graphicx, url, float, multicol}
\floatstyle{boxed}
\restylefloat{figure}

\title{Collabrador: a collaborative peer-to-peer text editor}
\newcommand{\name}[2]{#1 \\[-4pt] {\small \url{#2}} \\[4pt]}
\author{
  \name{Jacob Hurwitz}{jhurwitz@mit.edu}
  \name{Colleen Josephson}{cjoseph@mit.edu}
  \name{David Lawrence}{dlaw@mit.edu}}
\date{
  May 10, 2012 \\ \small
  R07 and R09 with Nir Shavit}
\begin{document}
\maketitle

\section{Introduction}

Collabrador is a peer-to-peer text editor that lets multiple users
collaborate on a document. Users can edit a document while online or
offline, although they cannot view each other's changes until they
join a (possibly ad-hoc) network and synchronize.  In addition to
storing the text of a document, Collabrador logs each user's edits as
individual insert, delete, and move operations.  When users make
distinct changes to a document, the branches can be merged by
replaying the changes made by one user atop the changes made by
another user.  This technique is known as the operational transform
and is the basis for all major collaborative editors.

\section{Design}

Users have the ability to edit documents offline. Once the user comes
back online Collabrador synchronizes with other online users and
merges the changes made while offline. Direct connectivity is also
supported, using ad-hoc networking and Bonjour to discover other
Collabrador users. When a user joins the internet (or an ad-hoc
network), his Collabrador client will integrate him into the network,
merging the latest commit into his local copy.

A checkpoint occurs automatically when a user saves, and is named by
hash of the document, provenance graph, and change log. Checkpoints
happen when a user fully connected, offline, or connected ad-hoc. The
hash for a checkpoint will be unique unless users make the same
changes in the same way, thus there will always be agreement as to
which document version corresponds to a version name. We will also
allow user-provided comments e.g. ``Ben's final draft'', however the
user is responsible for ensuring that these are useful.

Commits are a special type of checkpoint that a user manually
initiates. The commit process involves merging changes from all users
into one agreed-upon document. Merges are autonomous unless there has
been two or more edits at the same point, relative to a common
ancestor. This is called a conflict. If conflicts occurs, the machine
running the merge algorithm will ask the user to resolve them by
hand. Once each user's changes have been merged, all users must
approve the resulting document before the commit is finalized.

\subsection{System architecture}

On each computer, Collabrador consists of a \textbf{text editor} and a
\textbf{checkpoint database}. Rather than storing the final text,
Collabrador's text editor saves a description of the
\textbf{operations} performed by the user. These operations are
inserting a character at a given position (\texttt{INSERT}), deleting
a character at a given position (\texttt{DELETE}), and using cut/paste
to move a block of text from one position to another (\texttt{MOVE});
collectively, these operations form an \textbf{operational
  transformation} describing how the user modified the initial
document. When the user clicks the ``save'' button, Collabrador
creates a \textbf{checkpoint} containing this operational
transformation and saves it into the local \textbf{checkpoint
  database}. [TODO: cite OT paper] [TODO: think about whether the user
should explicitly click ``save'' or if this should happen in the
background]

To be more specific, the checkpoint database is a dynamic perfect hash
table of \textbf{checkpoint objects}, keyed by the SHA-1 hash of the
object. A checkpoint object is a data structure encapsulating the
details of the operational transformation performed by that edit, the
hashes of the edit's parent or parents, the hash of the unique
checkpoint to be used as the base of the operational transformation,
and the \textbf{user visibility list}, which is just the list of users
that have locally stored this checkpoint. We will assume that SHA-1
has no collisions, so two checkpoints will have the same hash if and
only if they perform the same changes to the same base from the same
parents.

When the system consists of a single computer, the checkpoint database
is just a linear progression of checkpoints. Adding in multiple
computers, a hypothetical central server keeping track of edits on all
computers would model the checkpoint database as a directed tree, with
a branching node occurring whenever two computers make distinct
edits. If two computers are allowed to synchronize with each other,
then branches of this tree can join together (creating nodes with two
parents instead of just one) and so the checkpoint database becomes a
directed acyclic graph instead. The challenge with Collabrador is to
maintain this directed acyclic graph in a distributed fashion because
there is no central server. Collabrador's solution is for each user to
store in its local checkpoint database only the portion of the graph
that it knows about while a synchronization process in the background
lets users constantly communicate with each other to share and
exchange parts of the graph.

At most times, a user's checkpoint database will be a graph with only
a single leaf node, representing the most recent checkpoint created by
the user. A database with just one leaf node is in the \textbf{merged}
state. When synchronizing with another user, that user's checkpoint
objects will be copied into the local checkpoint database, causing the
graph to have two leaf nodes. A database with two leaf nodes is in the
\textbf{unmerged} state. Then, as described in [TODO: cite the other
section], the Collabrador merge algorithm will find the least common
ancestor of these leaves, merge the two edits into a single
checkpoint, and then save this checkpoint into the database as a node
with two parents and an operational transformation relative to the
least common ancestor.

\subsection{Sync algorithm}

<<<<<<< HEAD
\textbf{Synchronization} is a process which causes two computers each with a merged checkpoint database to converge to the same merged checkpoint database. The computer that initiates the synchronization process is called the \emph{initiator}, and the other computer is called the \emph{provider}. At a high level, the synchronization process performs three steps:
\begin{enumerate}
\item The provider sends its checkpoint database to the initiator.
\item The initiator, which is now unmerged, locally performs the merge algorithm described in Section~\ref{sec:merge}.
\item The initiator sends its (now merged) checkpoint database to the initiator.
\end{enumerate}
One possible implementation of the synchronization process would be to have the provider send its entire checkpoint database to the initiator in the first step. However, it is usually the case that most of the provider's database is already known to the initiator. To minimize the amount of communication required, the provider only needs to send the checkpoint objects it has that the initiator does not have.

Let $I$ be the leaf node of the initiator's database, let $P$ be the leaf node of the provider's database, and let $C$ be any common ancestor of $I$ and $P$. There is guaranteed to be at least one common ancestor because, at the very least, the object representing the initial blank document is necessarily an ancestor of all checkpoint objects. Thus, the provider only needs to send the nodes ``between'' $C$ and $P$. To further minimize the amount of the database transmitted, it's ideal for $C$ to be the \emph{least} common ancestor of $I$ and $P$.

To find the least common ancestor in a distributed fashion, the provider performs the following algorithm: It sends $P$'s hash to the initiator and asks if this object was already in the initiator's checkpoint database. If it was, then this step of the synchronization process is complete. However, Collabrador also needs to update $P$'s user visibility list to include both the initiator and the provider (if they weren't already included). If $P$ was not already in the initiator's database, then the initiator asks for the full checkpoint object corresponding to $P$, and then the provider sends both this and recusively sends the hashes of $P$'s parents using the same process. This process will terminate upon reaching $C$. At this point, the initiator has the provider's entire checkpoint database.

Next, the initiator performs the merge algorithm using this most recently transmitted object $C$ as the common ancestor, and sets the user visibility list of the resulting leaf node to contain only the provider and the initiator. Finally, the initiator sends its entire checkpoint database to the provider using the aforementioned process.

When there are more than two computers, Collabrador has them synchronize in a pairwise fashion. The implementation of Collabrador assumes that each user can query for the set of all users currently on the network. Every computer has a background process that, at random intervals, sends a request to another computer asking if it is willing to synchronize. The two conditions under which a computer will refuse a request are if it is waiting for a response from any computer about synchronizing (up to some time-out), or if it is currently synchronizing. When two computers agree to synchronize, the computer that initiated the request will play the role of initiator and the other computer will be the provider. When this process finishes, the initiator will wait another random interval and then send a request to the user currently on the network with the next-lowest IP address in a round-robin process. After a quadratic number of pairwise synchronizations, all computers in the network will converge on the same checkpoint database.

Additionally, whenever two computers synchronize, they also exchange their \textbf{known user lists}. All users, whether online or offline, are included. The purpose of this exchange is so that all users collectively can determine the exact group membership, even if users are dynamically joining the group. This does not, however, support users dynamically leaving the group.
=======
The process for \textbf{synchronizing} the checkpoint databases of two
computers is straightforward. One computer, the \emph{initiator},
sends a request to the other computer, the \emph{provider}, asking for
the hash of the leaf node in its checkpoint database. If this
checkpoint is already in the initiator's database, then both the
initiator and provider are added to the checkpoint's user visibility
list (if they weren't already included), and then the synchronization
proceeds to the second stage of synchronization, in which the
initiator sends its entire database to the provider using a similar
process. Otherwise, the initiator asks for the full checkpoint object
of this leaf, and then recursively requests the checkpoint objects
corresponding to its parents until the initiator has incorporated the
provider's entire checkpoint database. At this point, the initiator
runs the merge algorithm, sets the resulting checkpoint's user
visibility list to contain only the initiator and provider, and then
proceeds to the second stage of synchronization.

When there are more than two computers, Collabrador has them
synchronize in a pairwise fashion. The implementation of Collabrador
assumes that each user can query for the set of all users currently on
the network. Every computer has a background process that, at random
intervals, sends a request to another computer asking if it is willing
to synchronize. The two conditions under which a computer will refuse
a request are if it is waiting for a response from any computer about
synchronizing (up to some time-out), or if it is currently
synchronizing. When two computers agree to synchronize, the computer
that initiated the request will play the role of initiator and the
other computer will be the provider. When this process finishes, the
initiator will wait another random interval and then send a request to
the user currently on the network with the next-lowest IP address in a
round-robin process. After a quadratic number of pairwise
synchronizations, all computers in the network will converge on the
same checkpoint database.

Additionally, whenever two computers synchronize, they also exchange
their \textbf{known user lists}. All users, whether online or offline,
are included. The purpose of this exchange is so that all users
collectively can determine the exact group membership, even if users
are dynamically joining the group. This does not, however, support
users dynamically leaving the group.
>>>>>>> 29538e0ed3056d84c89d4bec33bf29fdf5499c57

\subsection{Merge algorithm}
\label{sec:merge}

The merge algorithm is based on the operational transform (``OT''),
which is the conflict resolution technique used by all major
collaborative software (notably including Google Docs and
SubEthaEdit).  We model a single OT as a sequence of non-conflicting
``insert'', ``delete'', and ``move'' operations.  Note that every OT
defines a transformation on character indices (although this
transformation need not be injective or surjective), and all OTs are
invertible.  A full formal description of OTs is given in \cite{wave}.

OTs can interact in several ways, which are shown in figure
\ref{fig:ot}:
\begin{itemize}
\item Any number of OTs that are generated sequentially may be
  composed into a single OT representing the combined effect.  We can
  thereby express an OT that represents the transformation between any
  commit and its parent, regardless of how many steps the user took to
  effect this transformation.
\item We may use an OT, viewed as a transformation on character
  indices, to transform the indices of another OT.  This process will
  fail if and only if there are conflicting changes in the two OTs.
\item When two OTs are performed simultaneously, we may use one OT to
  transform the character indices of the second OT, and compose the
  result with the first OT.  This process is commutative when there
  are no conflicts.
\end{itemize}

When the merge algorithm takes over, the sync algorithm has copied
nodes between two merged DAGs, resulting in an unmerged DAG.  The sync
algorithm has also identified the lowest common ancestor of the two
leaves in the unmerged DAG.  We must merge these into a single
leaf---automatically if possible, but with user intervention if
necessary. Formally, given two OTs that take the same parent to
different children, we wish to generate a single OT that applies both
sets of changes to the parent in a logical way.

We simply follow the approach mentioned above: use one OT to transform
the character indices of the second OT, and compose the result with
the first OT.  Conflicting portions of the second OT are ignored and
flagged for manual user resolution.  (Following the New Jersey design
approach, we prohibit incorrect automatic merges, but allow
unnecessary manual merges if it makes the implementation substantially
simpler.)  When conflicting changes are identified, they are flagged
for manual resolution, but the algorithm continues to resolve
subsequent non-conflicting changes.  An example is shown in figure
\ref{fig:merge}.

\begin{figure}[h]
  \centering
  \begin{minipage}{\textwidth}
    \begin{multicols}{2}
      \setlength{\parskip}{-6pt}
      \subsubsection*{OT primitive operations}
      An OT is defined as a composition of the following primitive
      operations:
      \begin{eqnarray*}
        &insert(substring, index) \\
        &delete(start\_index, end\_index) \\
        &move(start\_index, end\_index, new\_loc)
      \end{eqnarray*}      
      \subsubsection*{Example strings}
      \begin{eqnarray*}
        A &=& \mathrm{^0i^1n^2s^3i^4d^5e^6\_^7o^8u^9t^{10}} \\
        B &=& \mathrm{^0o^1u^2t^3s^4i^5d^6e^7\_^8i^9n^{10}} \\
        C &=& \mathrm{^0o^1u^2t^3s^4i^5d^6e^7} \\
        D &=& \mathrm{^0i^1x^2n^3s^4i^5d^6e^7\_^8o^9u^{10}t^{11}}
      \end{eqnarray*}
      (Character indices are shown for clarity.)
      \subsubsection*{Example OTs}
      \begin{eqnarray*}
        X &=& move(5, 8, 0) \circ move(0, 2, 10) \\
        Y &=& delete(7, 10) \\
        Z &=& insert(1, \mathrm{x})
      \end{eqnarray*}
      We have \(X(A) = B\), \(Y(B) = C\), and \(Z(A) = D\).
      \subsubsection*{Composition}
      The composition \(Y \circ X\) is performed in the obvious way,
      and we have \((Y \circ X)(A) = C\).
      \subsubsection*{Parallel composition}
      When two users perform diverging modifications (for example,
      \(X\) and \(Z\) to \(A\)), we merge them by transforming the
      transform.  For example:
      \begin{eqnarray*}
        X(Z) &=& insert(9, \mathrm{x}) \\
        Z(X) &=& move(5, 8, 0) \circ move(0, 3, 11)
      \end{eqnarray*}
      We would merge the users' changes as follows: \[(X(Z) \circ
      X)(A) = (Z(X) \circ Z)(A) = \mathrm{outside\_ixn}.\] This
      process is formally described in \cite{ot}.
      \subsection*{Merge conflict}
      The quantities \((Y \circ X)(Z)\) and \(Z(Y \circ X)\) are not
      defined.  Both transformations of indices fail because there are
      conflicting changes.
    \end{multicols}
  \end{minipage}
  \caption{Operational transform examples.}
  \label{fig:ot}
\end{figure}

\begin{figure}[h]
  \centering
  \begin{minipage}{\textwidth}
    \begin{multicols}{2}
      \subsubsection*{The scenario}
      Start with the document ``abc'', and consider the following OTs:
      \begin{eqnarray*}
        U &=& insert(3, \mathrm{z}) \circ insert(1, \mathrm{x}) \\
        V &=& insert(1, \mathrm{y}) \circ insert(2, \mathrm{z})
      \end{eqnarray*}
      So \(U(\mathrm{abc}) = \mathrm{axbzc}\) and \(V(\mathrm{abc}) =
      \mathrm{aybzc}\).  The user must manually resolve the``x'' vs
      ``y'' merge conflict, but ``z'' should be merged automatically.
      \subsubsection*{Merge difficulties}
      Both users have made the same change (inserting ``z''), so a
      merge conflict should not be generated for that character.  The
      users also made a conflicting change (``x'' vs ``y''), but one
      did so before inserting ``z'' and one did so after inserting
      ``z''.
      \subsubsection*{The merge}
      First we compute \(U(V)\). \(insert(1,\mathrm{x})\) transforms
      \(insert(2,\mathrm{z})\) to \(insert(3,\mathrm{z})\).  We
      attempt to apply the coordinate transformation of
      \(insert(1,\mathrm{x})\) to \(insert(1,\mathrm{y})\) but this
      fails with a merge conflict.  We then construct \(U'\) and
      \(V'\) with the conflicting primitives removed, using the
      inverse transform to determine that \(U' = V' = insert(2,
      \mathrm{z})\).  Now we can compute the coordinate transformation
      \(U'(V')\): we transform \(insert(2, \mathrm{z})\) by
      \(insert(2, \mathrm{z})\).  Although the insertion coordinates
      conflict, since they insert the same string, we merge them into
      \(insert(2, \mathrm{z})\).  We apply this to ``abc'', resulting
      in ``abzc''.  Finally, we present the user with ``abzc'' along
      with the conflicting OTs (which have been transformed to the new
      coordinate space): \(insert(1,\mathrm{x})\) and
      \(insert(1,\mathrm{y})\).
    \end{multicols}
  \end{minipage}
  \caption{An especially difficult merge conflict.}
  \label{fig:merge}
\end{figure}

\subsection{Storage of the DAG}

The system used by clients to store the DAG of document versions must
meet performance and reliability requirements.

The performance requirements are easily addressed by having each node
store pointers to its children, parents, and the lowest common
ancestor of its parents.  Since nodes are named by hash, we also
maintain an index table containing pointers to nodes by hash.
Finally, we keep pointers to all current leaf nodes (usually one, but
sometimes two), and the original common ancestor for the document.
provide quick operations and fault tolerance.  Using this structure,
we may perform all relevant graph traversals in constant time per
node.

To meet our reliability (particularly atomicity) requirements, we
store the DAG in a database that is designed specifically to provide
fault-tolerance guarantees.  One such candidate is PostgreSQL
\cite{postgres}.

\subsection{Checkpoints}

In order to support named checkpoints, known as \textbf{commits},
Collabrador needs one additional data structure. Each user should
store a hash table mapping names to checkpoint hashes. With this data
structure, the algorithm for returning a specific named version of a
document is as easy as looking up the hash associated with the name,
and then looking up the checkpoint object associated with that hash.

The process for creating a commit is mostly the same as the process
for creating an unnamed checkpoint. The only difference is that,
during the synchronization process, users also exchange the name of
the checkpoint and its associated hash. If the initiator successfully
synchronizes with all users on the network, the initiator's leaf node
is the same hash as at the start of the process, and the leaf node's
user visibility list exactly equals the computer's known user list,
then the commit is successful. If any of these conditions are not met
within a pre-set number of passes through the synchronization
round-robin, the commit fails.

The specifications of the Collabrador system ensure that a successful
run of the commit algorithm guarantees correctness of the
process. Evidently, all users in the user visibility list have the
committed version and associated commit name stored
locally. Additionally, the initiator's user visibility list must
necessarily contain all members of the group (even ones who
dynamically joined) because the synchronization process added all
users' user visibility lists to the initiator's. The only possible
failure mode is if there is a new user who has never joined the same
network as a pre-existing user, but this new user can be safely
disregarded because it has never had the chance to obtain any version
of the document, much less modify it.

Finally, it is worth noting what happens in the case of a failed
commit. Imagine trying to commit a version named ``final'' and it
fails, and then trying again. If a user in the system is asked for the
version ``final'' before the second commit finishes, there is no
guarantee as to what version of the document it will return. To avoid
this problem, Collabrador requires that no two commit attempts use the
same name. Because some users may not know the names of all failed
commits, this restriction is enforced in practice, not in software. In
other words, the human user needs to know that trying the same name
twice can lead to undesirable results, but there are no software
safeguards to prevent this situation from occurring.

\section{Analysis}

\subsection{Design requirements}

TODO by colleen

\subsection{Performance}

TODO by dlaw and jhurwitz

\subsection{Drawbacks}

TODO by colleen

\section{Conclusion}

TODO by colleen

\bibliographystyle{IEEEtran} \bibliography{IEEEabrv,dp2}

TODO WORDCOUNT words

\end{document}
